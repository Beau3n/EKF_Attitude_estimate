\documentclass[10pt, a4paper]{article}
\usepackage{ucas_matrix}
\usepackage{tikz}

\chapter{航天器姿态确定算法与模型报告}

\section{引言}
本报告详细阐述了航天器姿态运动学与动力学模型,以及基于扩展卡尔曼滤波(Extended Kalman Filter, EKF)的姿态确定方法。报告结合了四元数运动学方程和陀螺仪误差模型,推导了用于高精度姿态估计的乘性扩展卡尔曼滤波(Multiplicative EKF, MEKF)算法。

\section{航天器姿态运动学与动力学模型}

\subsection{坐标系定义}
为了描述航天器的姿态,定义以下坐标系:
\begin{itemize}
    \item \textbf{惯性坐标系 (ECI, $i$-frame)}:原点位于地心,Z轴指向北天极,X轴指向春分点。
    \item \textbf{本体坐标系 (Body, $b$-frame)}:固连于航天器本体,原点位于航天器质心,坐标轴沿航天器主轴方向。
\end{itemize}

\subsection{姿态表示与运动学模型}
采用单位四元数 $q = [q_0, q_1, q_2, q_3]^T$ 描述本体坐标系相对于惯性坐标系的旋转。四元数满足归一化约束 $\|q\| = 1$。

四元数的微分运动学方程由航天器本体角速度 $\omega = [\omega_x, \omega_y, \omega_z]^T$ 驱动:
\begin{equation}
    \dot{q} = \frac{1}{2} q \otimes \omega = \frac{1}{2} \Omega(\omega) q
\end{equation}
其中,符号 $\otimes$ 表示四元数乘法。$\omega$ 为纯虚四元数 $[0, \omega^T]^T$。$\Omega(\omega)$ 是对应的 $4 \times 4$ 偏希得矩阵:
\begin{equation}
    \Omega(\omega) = \begin{bmatrix}
        0 & -\omega_x & -\omega_y & -\omega_z \\
        \omega_x & 0 & \omega_z & -\omega_y \\
        \omega_y & -\omega_z & 0 & \omega_x \\
        \omega_z & \omega_y & -\omega_x & 0
    \end{bmatrix}
\end{equation}

\subsection{姿态动力学模型}
航天器的姿态动力学描述了角速度随时间的变化,由欧拉方程(Euler's Equations)给出:
\begin{equation}
    I \dot{\omega} + \omega \times (I \omega) = T_{ext} + T_{ctrl}
\end{equation}
其中:
\begin{itemize}
    \item $I$ 为航天器的转动惯量矩阵。
    \item $\omega$ 为本体角速度。
    \item $T_{ext}$ 为外部干扰力矩(如重力梯度、气动力矩等)。
    \item $T_{ctrl}$ 为控制力矩(由飞轮或推力器产生)。
\end{itemize}
在姿态确定算法中,通常使用陀螺仪直接测量角速度 $\omega$,因此动力学方程主要用于仿真或过程噪声建模。

\subsection{传感器测量模型}

\subsubsection{陀螺仪模型}
陀螺仪测量输出 $\omega_{meas}$ 包含真实角速度 $\omega$、零偏 $\beta$ 和测量噪声 $\eta_v$:
\begin{equation}
    \omega_{meas} = \omega + \beta + \eta_v
\end{equation}
其中 $\eta_v$ 为角度随机游走(Angle Random Walk, ARW)噪声,满足高斯白噪声特性:$E[\eta_v(t)\eta_v^T(\tau)] = \sigma_v^2 I \delta(t-\tau)$。

陀螺仪零偏 $\beta$ 随时间缓慢漂移,通常建模为一阶马尔可夫过程或随机游走过程:
\begin{equation}
    \dot{\beta} = \eta_u
\end{equation}
其中 $\eta_u$ 为速率随机游走(Rate Random Walk, RRW)噪声,满足 $E[\eta_u(t)\eta_u^T(\tau)] = \sigma_u^2 I \delta(t-\tau)$。

\subsubsection{星敏感器模型}
星敏感器直接提供惯性系下的姿态四元数测量值 $q_{star}$。测量模型可以表示为真实姿态 $q_{true}$ 叠加一个小的测量噪声旋转 $\delta q_{meas}$:
\begin{equation}
    q_{star} = q_{true} \otimes \delta q_{meas}
\end{equation}
其中 $\delta q_{meas} \approx [1, \frac{1}{2}v_{noise}^T]^T$, $v_{noise}$ 为测量噪声角,服从零均值高斯分布 $R$。

\section{姿态确定方法:乘性扩展卡尔曼滤波 (MEKF)}

\subsection{算法概述}
乘性扩展卡尔曼滤波(MEKF)利用误差状态四元数来避免协方差矩阵奇异性,并严格保持四元数的归一化约束。算法分为\textbf{状态}和\textbf{误差状态}两部分。

\textbf{状态向量}定义为:
\begin{itemize}
    \item $q$: 全局姿态四元数(Nominal State)。
    \item $\beta$: 陀螺仪零偏。
\end{itemize}

\textbf{误差状态向量} $\delta x$ (6维) 定义为:
\begin{equation}
    \delta x = \begin{bmatrix} \delta \theta \\ \delta \beta \end{bmatrix}
\end{equation}
其中 $\delta \theta$ 为本体坐标系下的姿态误差角,$ \delta \beta = \beta_{true} - \hat{\beta}$ 为零偏误差。

姿态误差 $\delta q$ 定义为真实姿态与估计姿态之间的相对旋转(在本体坐标系下):
\begin{equation}
    q_{true} = \hat{q} \otimes \delta q \approx \hat{q} \otimes \begin{bmatrix} 1 \\ \frac{1}{2}\delta \theta \end{bmatrix}
\end{equation}

\subsection{时间更新(预测步骤)}

\subsubsection{1. 状态预测}
利用陀螺仪测量 $\omega_{meas}$ 进行姿态积分。
估计的角速度:
\begin{equation}
    \hat{\omega} = \omega_{meas} - \hat{\beta}_{k}
\end{equation}
四元数更新(离散化):
\begin{equation}
    \hat{q}_{k+1}^- = \hat{q}_k^+ \otimes \Delta q(\hat{\omega} \Delta t)
\end{equation}
其中 $\Delta q$ 由罗德里格斯公式计算:
\begin{equation}
    \Delta q = \begin{bmatrix} \cos(\|\hat{\omega}\|\Delta t / 2) \\ \frac{\hat{\omega}}{\|\hat{\omega}\|} \sin(\|\hat{\omega}\|\Delta t / 2) \end{bmatrix}
\end{equation}
零偏预测假设为常值:
\begin{equation}
    \hat{\beta}_{k+1}^- = \hat{\beta}_k^+
\end{equation}

\subsubsection{2. 误差协方差预测}
需要推导误差状态方程 $\delta \dot{x} = F \delta x + G w$。

\textbf{姿态误差动力学}:
对 $q_{true} = \hat{q} \otimes \delta q$ 求导并忽略高阶项,可得:
\begin{equation}
    \delta \dot{\theta} = -[\hat{\omega} \times] \delta \theta - \delta \beta - \eta_v
\end{equation}
其中 $[\hat{\omega} \times]$ 为反对称矩阵。

\textbf{零偏误差动力学}:
\begin{equation}
    \delta \dot{\beta} = \eta_u
\end{equation}

连续时间系统矩阵 $F(t)$ 为:
\begin{equation}
    F = \begin{bmatrix}
        -[\hat{\omega} \times] & -I_{3 \times 3} \\
        0_{3 \times 3} & 0_{3 \times 3}
    \end{bmatrix}
\end{equation}

离散化各状态转移矩阵 $\Phi_k \approx I + F \Delta t$。
协方差预测方程:
\begin{equation}
    P_{k+1}^- = \Phi_k P_k^+ \Phi_k^T + Q_k
\end{equation}
其中 $Q_k$ 为离散过程噪声协方差矩阵。

\subsection{量测更新(校正步骤)}

\subsubsection{1. 测量残差计算}
星敏感器测量 $q_{star}$ 与预测姿态 $\hat{q}^-$ 之间的偏差。
定义测量残差四元数 $\delta q_{err}$ (注意顺序依赖于定义,本报告采用代码中的逻辑):
\begin{equation}
    \delta q_{err} = q_{star} \otimes (\hat{q}^-)^{-1}
\end{equation}
提取矢量部分作为观测变量 $z$(小角度假设):
\begin{equation}
    z = 2 \cdot \text{vec}(\delta q_{err}) \approx \delta \theta_{meas}
\end{equation}

\subsubsection{2. 观测矩阵与卡尔曼增益}
由于 $z$ 直接对应姿态误差 $\delta \theta$,且不包含零偏信息,观测矩阵 $H$ 为:
\begin{equation}
    H = \begin{bmatrix} I_{3 \times 3} & 0_{3 \times 3} \end{bmatrix}
\end{equation}

计算卡尔曼增益 $K$:
\begin{equation}
    S = H P^- H^T + R
\end{equation}
\begin{equation}
    K = P^- H^T S^{-1}
\end{equation}

\subsubsection{3. 状态与协方差更新}
计算误差状态估计值:
\begin{equation}
    \delta \hat{x} = [\delta \hat{\theta}^T, \delta \hat{\beta}^T]^T = K z
\end{equation}

\textbf{协方差更新}(使用Joseph形式以保证数值稳定性):
\begin{equation}
    P^+ = (I - KH) P^- (I - KH)^T + K R K^T
\end{equation}

\textbf{全状态注入(Reset)}:
利用计算出的误差项修正名义状态。
姿态修正:
\begin{equation}
    \hat{q}^+ = \hat{q}^- \otimes \delta q(\delta \hat{\theta}) \approx \hat{q}^- \otimes \begin{bmatrix} 1 \\ \frac{1}{2}\delta \hat{\theta} \end{bmatrix}
\end{equation}
修正后需对 $\hat{q}^+$ 进行归一化。

零偏修正:
\begin{equation}
    \hat{\beta}^+ = \hat{\beta}^- + \delta \hat{\beta}
\end{equation}

\section{结论}
上述 MEKF 算法结合了陀螺仪的高频动态特性和星敏感器的低频高精度特性,通过误差状态的高效迭代,能够实现航天器姿态的精确估计,并同时在线估计和补偿陀螺仪漂移。

